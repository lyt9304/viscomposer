\section{Introduction}
\label{sec:intro}
As the amount of data being collected has increased, the need for tools that can enable the visual exploration of data has also grown.  This has led to the development of a variety of widely used programming frameworks for information visualization~\cite{Flare,Processing,Heer:2009:TVCG,Heer:2011:TVCG,Heer:2010:TVCG,Heer:2005:CHI}.  Unfortunately, such frameworks demand comprehensive visualization and coding skills and require users to develop visualization from scratch. As such, these visualization programming toolkits and languages require a large amount of effort for developing appropriate visualization solutions, thereby making visualization nearly intractable for non-programming experts.  Furthermore, conventional programming toolkits or languages seldom integrate with a What You See Is What You Get editor. Thus, they rarely support interactive configurations, thereby stifling collaboration between coders and visual artists in a shared and unified environment.

As such, a recent trend in information visualization is focused on creating interactive visualization design environments that require little to no programming~\cite{Wijk:2011:TVCG,Victor}.  Tools such as Lyra~\cite{Heer:2014:CGF} and iVisDesigner~\cite{Yuan:2014:TVCG} have been created as visualization production tools that allow users to create sophisticated layouts and transformations that are enabled via transformation pipelines.  Unfortunately, these visualization production tools offer support for only a small portion of visual forms, thereby greatly limiting the design space.  As such, expressive visualizations for multivariate and heterogeneous datasets that can easily be created with visualization languages or toolkits like D3~\cite{Heer:2011:TVCG} and Processing~\cite{Processing}, can be difficult to realize in many current visualization production tools.  For instance, Lyra~\cite{Heer:2014:CGF} only operates on tabular datasets, and iVisDesigner~\cite{Yuan:2014:TVCG}, while being very efficient at generating coordinate systems, does not support the construction of recursive drawings such as treemaps.  Furthermore, specifying customized visual designs in these tools is only feasible when the interaction workload is moderate, such as mapping selected data dimensions to appropriate visual channels, or modulating the color scheme for a group of selected data items. The task of managing and customizing detailed designs on a large data set becomes increasingly complicated or even intractable.

Unfortunately, the extremes of a fully programmable solution and a purely interactive non-programmable design environment have major shortcomings. In this paper, we propose a hybrid solution in the form of a novel programmable integrated development environment (IDE), VisComposer.  Such hybrid solutions have a long history in the graphics community.  For example, with the development of shading languages~\cite{RenderMan,OpenGLSL} and programmable graphics hardware, shader-based programs replaced the traditional fixed rendering pipeline to achieve more comprehensive effects~\cite{RTVG,Rieder:2011:CGF}.  To further improve performance and flexibility, many interactive shader development environments, such as RenderMonkey~\cite{RenderMonkey} and FX Composer~\cite{FXComposer}, were created to provide graphic artists with an IDE that served as both a production tool and a programmable interface.

VisComposer has been designed with the goal of making visualization design and optimization easier by providing an intuitive user interface to customize effects with rich authoring controls. Our hybrid solution improves on previous work by enlarging the design space of visualization available to the user.  This is enabled through a custom scene graph in which every component of the visualization process can be configured, edited, and shared.  To facilitate this process, we modularize the visualization design process into components that can be manipulated individually and composed to ensure the efficient, customized, and synchronized production of information visualization. We also adopt the design concepts of shaders from computer graphics in the context of information visualization, where each node in our scene graph can be defined as a piece of program that creates a specific visualization effect. VisComposer  not only provides a drag-and-drop visual editor for rapid prototyping of data-driven visualization, but also enables customization of special effects through textual programming. Our system compares favorably with existing IDEs~\cite{Yuan:2014:TVCG,Heer:2014:CGF} in that it allows for pre-defined visualization creation, while enhancing this through programmable operations, thereby creating a visualization composition language and a visual composer that supports the easy integration of various visualization components, as demonstrated with several examples shown in Figure~\ref{teaser}. Our contributions include:
\begin{itemize}
\item A visualization composition model that modularizes the visualization design process and abstracts the resultant visualization as a scene graph, in which nodes and edges are editable and programmable.
\item A visualization composer that provides a set of operations to enable intuitive and programmable visualization design. Its implementation is built upon the JavaScript language and is compatible with JavaScript-based visualization programs, e.g. D3.js~\cite{Heer:2011:TVCG}.
\item An integrated visual composition environment that empowers users with the ability to directly program visualization effects and immediately view the results of their programming operations.  In this way, we can reduce the time needed for development while still providing the flexibility for novel visualization design.  Such an environment enables the quick iteration of novel designs that cannot be encapsulated in a non-programmable environment.
\end{itemize}





