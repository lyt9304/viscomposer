\section{Related Work}
\label{s:relatedwork}

\noindent \textbf{Visualization Programming Frameworks and Languages}
Over the years, a variety of frameworks and languages have been developed with varying degrees of uptake~\cite{Haeberli:1988:SIG,Wilkinson:Grammar}.  The Infovis toolkit~\cite{Fekete:2004:VIS} is an interactive graphics toolkit written in Java that provides a large set of visualization components and supports fast dynamic queries. A distinctive feature of the Infovis toolkit is that it enables many user interactions, e.g. fisheye lenses, dynamic labeling, etc. Improvise~\cite{Weaver:2004:VIS} is another attempt to provide developers with a fully-implemented Java software architecture with a declarative visual query language. It enables users to build and browse highly-coordinated visualizations through a visual interface. The focus of Improvise is on user-driven exploration of complicated datasets across multiple views. Flare~\cite{Flare} is an ActionScript library for creating visualizations that runs in the Adobe Flash Player. It supports data management, visual encoding, animation, and interaction of a list of visual forms. Prefuse~\cite{Heer:2005:CHI} is a software framework for producing dynamic visualizations with structured or unstructured data. The abstractions provided by Prefuse follow the data visualization pipeline proposed in Card et al.~\cite{Card:1999}. Processing~\cite{Processing} is a programming language that was initially designed as a software sketchbook to teach computer programming fundamentals within a visual context. Due to its simplicity, it is widely used by programmers, designers and visual artists. ProtoVis~\cite{Heer:2009:TVCG,Heer:2010:TVCG} is a declarative, domain-specific language for constructing interactive visualizations across platforms. A scenegraph is employed to organize the visual design. D3.js~\cite{Heer:2011:TVCG} inherits the basic abstractions from ProtoVis and improves the browser compatibility by directly binding the input data to standard document object model (DOM).

Each of these frameworks and languages has enabled analysts with varying degrees of expertise to create and share interactive data visualizations.  The power of these languages is in their extendability to create both traditional and novel visualizations.  The ubiquity of visualizations created by these libraries shows that there is a need for these visualization programming languages.  However, researchers have also recognized that there is a large overhead in using these programming languages to create visualizations from scratch.

\noindent \textbf{Integrated Development Environments for Visualization} The need for a faster visualization development time has led to the development of integrated environments for visualization~\cite{Silva:2005:VIS,Lee:2013:TVCG,Myers:1994:CHI}.  Commercial information visualization software and tools, such as Tableau and Manyeyes~\cite{Viegas:2007:TVCG}, provide the user with a drag-and-drop interface for creating visualizations from the scratch. However, the resultant visualization designs are typically adopted from canned visual forms, making it difficult (or in some cases impossible) to create novel customized visualization or effects.

Recent work has sought to improve the user experience and allow for a variety of expressive visualization creations.  For example, Lyra~\cite{Heer:2014:CGF} seeks to improve the expressiveness  by mapping the conventional data visualization  pipeline into a visual editor. The input data is interactively bound to the properties of graphical marks, and the author can design a new visualization which is represented as a specification in Vega~\cite{Vega} which then enables  the sharing and reuse of the visualization product. Similarly, iVisDesigner~\cite{Yuan:2014:TVCG} supports the interactive design of expressive visualization for heterogeneous datasets, covering a broad range of the information visualization design space.
Other tools include Ellipsis~\cite{Heer:2014:Ellipsis}, which implements a model of storytelling abstractions and a domain-specific language (DSL)  within a graphical interface for effective story authoring.

While these systems advance previous solutions by enabling visualization customization without writing code, the design space available to the visualization authors is still limited. On the other hand, the language based approach of building each visualization from scratch also has considerable drawbacks.  As such, it seems natural to integrate programming capability within an IDE for Visualization.  VisComposer presents a hybrid solution combining options for programmability into an IDE for visualization.  By enabling users to directly add code segments for customizable and extensible visualizations, our work is able to reduce the programming overhead common among visualization languages while still providing much of their flexibility.


One main advantage of integrating programmability into the graphical development environments is its flexibility and expressiveness. For example, in computer graphics and scientific visualization, shader programs~\cite{RTVG,OpenGLSL} are typically written to apply transformations to a large set of elements, e.g., to each pixel in a window. This is very suitable for parallel processing, and most GPUs have multiple shader pipelines to facilitate this parallelism, vastly improving computation throughput. Our approach shares the same benefit in that a visualization operation written within the interactive visualization environment can be applied to many data points, avoiding individual specifications of the intended operations.
